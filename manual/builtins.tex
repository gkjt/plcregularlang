\section{Builtin Functions}
\label{builtins}
\begin{normalsize}
LangLang provides a number of built in functions, detailed below. Many of these functions can be nested without brackets, although we recommend using brackets to keep the meaning of the program clear.
\end{normalsize}

\subsection{Print}
\label{print}
\begin{center}
\begin{minipage}{8cm}
\begin{verbatim}
print OBJECT (optional LIMIT) -> unit
\end{verbatim}
\end{minipage}
\end{center}
\begin{normalsize}
Prints a representation of OBJECT to STDOUT. If OBJECT is a language, LIMIT can be used to print only a number of elements from the language. This does not affect the underlying data. All print calls end with a newline.

See section \ref{helloworld} for an example of using print.
\end{normalsize}

\subsection{Union}
\begin{center}
\begin{minipage}{5cm}
\begin{verbatim}
LANG U LANG -> language
\end{verbatim}
\end{minipage}
\end{center}
\begin{normalsize}
Return the union of two languages.
\end{normalsize}

\subsection{Intersection}
\begin{center}
\begin{minipage}{5cm}
\begin{verbatim}
LANG I LANG -> language
\end{verbatim}
\end{minipage}
\end{center}
\begin{normalsize}
Returns the intersection of two languages. Intersection is left and right associative, so calls can be chained into one line with no issue.
\end{normalsize}

\subsection{Concatenation}
\begin{center}
\begin{minipage}{5cm}
\begin{verbatim}
FIRST + SECOND -> language
\end{verbatim}
\end{minipage}
\end{center}
\begin{normalsize}
Concatenates every item of the language FIRST with every item of SECOND. Either parameter may be a string, which is treated as a single word set. This operation is not commutative, but is associative and may be safely chained.
\end{normalsize}

\subsection{Powerset}
\begin{center}
\begin{minipage}{5cm}
\begin{verbatim}
LANG ^ LIMIT -> language
\end{verbatim}
\end{minipage}
\end{center}
\begin{normalsize}
Returns the set consisting of all elements of the language LANG multiplied by each other a LIMIT number of times.
\begin{center}
\begin{minipage}{5cm}
\begin{verbatim}
print {"a", "b", "c"} ^ 2;
\end{verbatim}
\end{minipage}
\end{center}
Outputs: $ \{aa, ab, ac, ba, bb, bc, ca, cb, cc\} $
\end{normalsize}

\subsection{Kleene's Star}
\begin{center}
\begin{minipage}{6cm}
\begin{verbatim}
ALPHABET * LENGTH -> language
\end{verbatim}
\end{minipage}
\end{center}
\begin{normalsize}
Returns the set of elements of length LENGTH from the language over the set of symbols ALPHABET, which is also a language.

\begin{center}
\begin{minipage}{5cm}
\begin{verbatim}
print {"a", "b", "c"} * 5;
\end{verbatim}
\end{minipage}
\end{center}
Outputs: $ \{:, a, aa, aaa, aaaa\} $

All elements of ALPHABET must be single characters, passing anything longer will cause it to only use the first character of each element in all but the first non empty element.
\end{normalsize}

\subsection{Read Integer}
\begin{center}
\begin{minipage}{5cm}
\begin{verbatim}
readint -> integer
\end{verbatim}
\end{minipage}
\end{center}
\begin{normalsize}
Gets the next item from the input buffer if it is an integer, else will throw an error.
\end{normalsize}

\subsection{Read Language}
\begin{center}
\begin{minipage}{5cm}
\begin{verbatim}
readlang -> language
\end{verbatim}
\end{minipage}
\end{center}
\begin{normalsize}
Gets the next item from the input buffer if it is a language, else will throw an error.
\end{normalsize}




\iffalse
\subsection{Addition}
\begin{center}
\begin{minipage}{5cm}
\begin{verbatim}
LANG I LANG -> 
\end{verbatim}
\end{minipage}
\end{center}
\begin{normalsize}

\end{normalsize}
\fi
