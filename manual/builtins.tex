\section{Builtin Functions}
\label{builtins}
\begin{normalsize}
LangLang provides a number of built in functions, detailed below. Many of these functions can be nested without brackets, although we recommend using brackets to keep the meaning of the program clear.
\end{normalsize}

\subsection{Print}
\label{print}
\begin{verbatim}
print OBJECT (optional LIMIT) -> unit
\end{verbatim}
\begin{normalsize}
Prints a representation of OBJECT to STDOUT. If OBJECT is a language, LIMIT can be used to print only a number of elements from the language. This does not affect the underlying data. All print calls end with a newline.

See section \ref{helloworld} for an example of using print.
\end{normalsize}

\subsection{Union}
\begin{verbatim}
LANG U LANG -> language
\end{verbatim}
\begin{normalsize}
Return the union of two languages
\end{normalsize}

\subsection{Intersection}
\begin{verbatim}
LANG I LANG -> language
\end{verbatim}
\begin{normalsize}
Returns the intersection of two languages. Intersection is left and right associative, so calls can be chained into one line with no issue.
\end{normalsize}

\subsection{Concatenation}
\begin{verbatim}
FIRST + SECOND -> language
\end{verbatim}
\begin{normalsize}
Concatenates every item of the language FIRST with every item of SECOND. Either parameter may be a string, which is treated as a single word set. This operation is not commutative, but is associative and may be safely chained.
\end{normalsize}

\subsection{Addition}
\begin{verbatim}
LANG + LANG  -> int
\end{verbatim}
\begin{normalsize}

\end{normalsize}

\subsection{Powerset}
\begin{verbatim}
LANG ^ LIMIT -> language
\end{verbatim}
\begin{normalsize}

\end{normalsize}

\subsection{Kleene's Star}
\begin{verbatim}
ALPHABET * LENGTH -> language
\end{verbatim}
\begin{normalsize}
Returns the set of elements of length LENGTH from the language over the set of symbols ALPHABET, which is also a language. There are no constraints on what may be in ALPHABET, but be aware that each element of the alphabet is counted as length 1, so:
\begin{verbatim}
print {"abc"} * 1;
\end{verbatim}
outputs $ abc $
\end{normalsize}

\subsection{Read Integer}
\begin{verbatim}
readint -> integer
\end{verbatim}
\begin{normalsize}
Gets the next item from the input buffer if it is an integer, else will throw an error.
\end{normalsize}

\subsection{Read Language}
\begin{verbatim}
readlang -> language
\end{verbatim}
\begin{normalsize}
Gets the next item from the input buffer if it is a language, else will throw an error.
\end{normalsize}




\iffalse
\subsection{Addition}
\begin{verbatim}
LANG I LANG -> 
\end{verbatim}
\begin{normalsize}

\end{normalsize}
\fi