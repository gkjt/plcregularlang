\subsection{Primitive Types}
\begin{normalsize}
In LangLang there are 3 primitive types: \textit{strings}, \textit{languages}, and \textit{integers}. This limits the scope of the language, but makes it very simple to work with, with very few concepts to understand before you can use the entire language.
\end{normalsize}
\subsubsection{Integers}
\begin{normalsize}
LangLang has one number type, the integer. It holds a decimal number in the range $ 0 ... 2^{62}-1 $.

Example:
a = 12345;
\end{normalsize}
\subsubsection{Strings}
\begin{normalsize}
Strings in LangLang are surrounded by double quotes and contain lower case English letters, and can only be defined inside the program.

Example:
a = "youjustlostthegame";

\end{normalsize}
\subsubsection{Languages}
\begin{normalsize}
Languages are possibly the most important data type in LangLang. These are representations of finite size regular languages of finite length strings. These can be defined either in the input or the program itself
\end{normalsize}